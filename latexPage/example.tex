% !Mode:: "TeX:UTF-8"
%!TEX program  = xelatex

% \documentclass{cumcmthesis}
\documentclass[withoutpreface,bwprint]{cumcmthesis} %去掉封面与编号页
\usepackage{url}
\usepackage{graphicx}
\usepackage{float}
\usepackage{booktabs} 
\usepackage{threeparttable}
\newcommand{\upcite}[1]{\textsuperscript{\textsuperscript{\cite{#1}}}}
\usepackage[utf8]{inputenc}

% 设置页眉线
\usepackage{fancyhdr}
\fancyhf{}
\pagestyle{fancy}
\rhead{\textcolor{black}{参赛队伍编号:}\textcolor{red}{TFB0000404SXJM}}

\renewcommand{\headrulewidth}{2pt}
\renewcommand{\headrule}{\color{red}\hrule width\headwidth height\headrulewidth \vskip-\headrulewidth}

\title{题\quad 目:\underline{\quad\quad 基于。。。 \quad\quad}}

% \tihao{A}
% \baominghao{4321}
% \schoolname{XX大学}
% \membera{zstar}
% \memberb{向左}
% \memberc{哈哈}
% \supervisor{老师}
% \yearinput{2021}
% \monthinput{08}
% \dayinput{22}

\begin{document}

% 摘要
\maketitle
\begin{abstract}
	摘要

	\keywords{1;\quad 1;\quad 1;\quad }
\end{abstract}

%目录(可要可不要)
% \tableofcontents

\section{问题重述}

\subsection{背景资料}

地震的

\subsection{需要解决的问题}

我们通过分析相关数据,运用数学思想,建立数学模型来研究震源识别、震级预测中的下列问题:

问题1: 结合

问题2: 已知

问题3: 水库

\section{问题分析}

\subsection{问题一分析}

问题一分析


\subsection{问题二分析}


\subsection{问题三分析}


\section{模型的假设}

\begin{enumerate}
	\item 假设地震波的真实数据与噪声数据存在显著差异。
	\item 假设各站点检测仪器准确,不考虑仪器误差。
	\item 假设不同的地震事件直接互不干扰。
\end{enumerate}

\section{符号说明}

\begin{center}
	\begin{tabular}{cc}
		\hline
		\makebox[0.3\textwidth][c]{符号} & \makebox[0.4\textwidth][c]{意义} \\
		\hline
		$\otimes$                      & 卷积运算                           \\
		$x(t)$                         & 原始信号                           \\
		$\psi(t)$                      & 小波基函数                          \\
		$a$                            & 尺度因子                           \\
		\hline
	\end{tabular}
\end{center}

% 问题一
\section{问题一的建立与求解}

\subsection{统计分析}

% 正文



%%%%%%%%%%%%%%%%%%%%%%%%%%%%%%%%%%%%%%%%%%%%%%%%%%%%%%%%

%参考文献
\bibliographystyle{unsrt}
\bibliography{reference}

\end{document}